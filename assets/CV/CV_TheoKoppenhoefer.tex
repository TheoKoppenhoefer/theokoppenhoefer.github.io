% This template was taken from https://bitbucket.org/bklamer/resume/src/master/minimal-bullet-resume.tex

% Compile using LuaLaTeX
\documentclass[11pt, letterpaper]{article}

% \usepackage{fontspec}
\usepackage[oldstylenums]{kpfonts}
\usepackage[T1]{fontenc}
\usepackage[left=0.65in, right=1in, vmargin=0.75in]{geometry}
\usepackage[final]{microtype}
\usepackage{tabularx}
\usepackage{titlesec}
\usepackage{enumitem}
\usepackage[colorlinks=true,urlcolor=black]{hyperref}

\setlength{\parindent}{0pt}
\titleformat{\section}{\Large\scshape\raggedright}{}{0em}{}
\pagenumbering{gobble}

\begin{document}

%==============================================================================
% "business card" section
%==============================================================================
\begin{minipage}[t][1.25in][t]{.5\textwidth}
\begin{tabular}{@{}l@{}} % @{} removes tabular outer padding
\\
{\Huge\scshape Theo Koppenhöfer}\\[2.35ex]
{\large Curriculum vitae}
\end{tabular}
\end{minipage}
\hfill
\begin{minipage}[t][1.25in][t]{.4\textwidth}
\hfill
\begin{tabular}{l@{}}
EH12 7BQ Edinburgh, United Kingdom\\[1ex]
email: \href{mailto:tbk3000@hw.ac.uk}{tbk3000@hw.ac.uk} \\
web: \href{https://theokoppenhoefer.github.io/}{theokoppenhoefer.github.io/} \\
github: \href{https://github.com/TheoKoppenhoefer}{TheoKoppenhoefer}
\end{tabular}
\end{minipage}

%==============================================================================
% detail section
%==============================================================================
\section{About me}

Currently I am a PhD student at Heriot Watt University in Edinburgh.
You may find that I like maths. More specifically I focus on analysis of partial differential equations and more specifically water waves under the supervision
of \href{https://researchportal.hw.ac.uk/en/persons/daniel-coutand/}{Daniel Coutand}
and \href{https://researchportal.hw.ac.uk/en/persons/beatrice-pelloni/}{Beatrice Pelloni}.
During my bachelors I also focused on numerical analysis though I have taken courses all over the shop in mathematics (and some physics).
As a result of a summer project in 2024 I am one of the co-founders of \href{https://www.kravianalytics.com/}{kravianalytics}.
When I find the time I enjoy reading. A lot. Besides of that I regularly go out for hikes and cycle rides in the surrounding nature.
Originally I am from Heidelberg, Germany.

\section{Facts}
\begin{tabularx}{\textwidth}{>{\raggedright\arraybackslash}p{45mm} X}
  Full name & Theo Benjamin Koppenhöfer \\[0.5ex]
  Date of birth & 9. November 2000 \\[0.5ex]
  Place of birth & Heidelberg, Germany \\[0.5ex]
  Nationality & German, British
\end{tabularx}\section{Education}

\begin{tabularx}{\textwidth}{>{\raggedright\arraybackslash}p{45mm} >{\raggedright\arraybackslash}X}
autumn 2024\linebreak\hspace*{1em} -- 2027 & Mathematics PhD, \href{https://www.maxwell.ac.uk/}{Heriot Watt University}, Scottland \\[0.5ex]
  autumn 2022\linebreak\hspace*{1em} -- spring 2024 & Mathematics Msc., \href{https://maths.lu.se/}{Lunds Universitet}, Sweden \\[0.5ex]
  autumn 2023\linebreak\hspace*{1em} -- spring 2024 & Svenska som andraspråk 2, Komvux Lund, Sweden \\[0.5ex]
  autumn 2019\linebreak\hspace*{1em} -- summer 2022 & Mathematics Bsc., \href{http://www.math.uni-bonn.de/}{Universität Bonn}, Germany \\[0.5ex]
  autumn 2011\linebreak\hspace*{1em} -- summer 2019 & Abitur, \href{https://www.srgh.de/}{St. Raphael Gymnasium Heidelberg}, Germany
\end{tabularx}

\section{Theses}
\begin{tabularx}{\textwidth}{p{45mm} >{\raggedright\arraybackslash}X}
  Master thesis & \href{https://raw.githubusercontent.com/TheoKoppenhoefer/master-thesis/main/Text/Thesis_TheoKoppenhoefer.pdf}{Some relations between equilibria of harmonic vector fields and the domain topology},
  applied analysis, under the supervision of \href{https://www.maths.lu.se/english/research/staff/erik-wahlen/}{Erik Wahlén} \\
  Bachelor thesis & \href{https://raw.githubusercontent.com/TheoKoppenhoefer/bachelorarbeit/main/Text/Bachelorarbeit_Hauptteil.pdf}{Adaptive finite element methods in linear elasticity} (in German,  \linebreak mark: 1.1),
  numerics, under the supervision of \href{https://ins.uni-bonn.de/staff/gedicke}{Joscha Gedicke}
\end{tabularx}

\section{Computing skills}
\begin{tabularx}{\textwidth}{p{45mm} X}
  General programming & python, C, C++, OpenMPI \\ [1ex]
  Mathematical languages & Wolfram Matematica, Matlab, Maple \\ [1ex]
  Basic webdesign & HTML, Javascript, CSS, php (I maintained the website calcfee.com for a year where one could calculate paypal transaction fees) \\ [1ex]
  Linux / Unix & git, bash, makefiles \\ [1ex]
  Other & \LaTeX
\end{tabularx}

\section{Language skills}
\begin{tabularx}{\textwidth}{p{45mm} X}
  English & Native speaker \\ [1ex]
  German & Native speaker \\ [1ex]
  Swedish & Intermediary knowledge \\ [1ex]
  French & Basic knowledge (B1) \\ [1ex]
  Latin & Basic knowledge (Latinum)
\end{tabularx}

\section{Other}
\begin{tabularx}{\textwidth}{>{\raggedright\arraybackslash}p{45mm} >{\raggedright\arraybackslash}X}
October 2025\linebreak\hspace*{1em} -- January 2026 & PhD representative for the Heriot-Watt \href{https://www.hw.ac.uk/about/professional-services/governance-and-legal-services/policy-and-governance/senate/senate-committees/university-committee-for-research-and-innovation}{University Committee for Research and Innovation} \\[0.5ex]
  August 2025\linebreak\hspace*{1em} --  & PhD representative for mathematics at \href{https://www.maxwell.ac.uk/}{Heriot Watt University} \\[0.5ex]
  December 2024\linebreak\hspace*{1em} --  & One of the organisers of the \href{https://sites.google.com/view/hwmacsphdseminar/home/}{Heriot-Watt MACS PhD seminar} \\[0.5ex]
  Summer 2024\linebreak\hspace*{1em} --  & Spend my free time advising and coding for \href{https://www.kravianalytics.com/}{kravianalytics} \\[0.5ex]
  June 2024\linebreak\hspace*{1em} -- September 2024 & Programmed a prototype vessel detection software for what became \href{https://www.kravianalytics.com/}{kravianalytics}
\end{tabularx}

\end{document}

